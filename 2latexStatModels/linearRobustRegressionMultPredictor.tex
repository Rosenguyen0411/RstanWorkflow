\documentclass[12pt]{article}

\usepackage[top=1in, bottom=1in, left=1in, right=1in]{geometry}
\usepackage{amsfonts}
\usepackage{amsmath}
\usepackage{amssymb}

\setlength\parindent{0pt}
\begin{document}

\title{Probability Model For A Multi-Predictor Linear Robust Regression Model}
\author{Nan Wu \\ nanw@udel.edu}
\date{}
\maketitle

\section{Linear Regression Model}

The linear regression model is given as
\begin{equation}
y=\alpha+X\beta+\epsilon
\end{equation}

where $y\in \mathbb{R}^N$, $X\in \mathbb{R}^{N \times P}$, $\alpha \in \mathbb{R}$, $\beta \in \mathbb{R}^P$, $\epsilon \in \mathbb{R}^N$, and $\epsilon \sim \mathrm{t}\left( \nu,0,\sigma \right)$.

In this model, $X$ is the predictor matrix, $y$ is the outcome, $\alpha$ is the intercept, $\beta$ is the coefficient vector associated with predictors, $\epsilon$ is the residual following a students' t distribution.

\section{Probability Model}

Stan allows us to use improper priors if we don't have any prior knowledge about the parameters. We can therefore start with a simple model by assuming the following priors for $\alpha$, $\beta$, $\nu$ and $\sigma$:
\begin{align*}
\alpha &\sim \mathrm{Uniform}\left( -\infty,\infty \right) \\
\beta_p &\sim \mathrm{Uniform}\left( -\infty,\infty \right), p=1,\cdots,P \\
\nu &\sim \mathrm{Cauchy}_+\left(7,5 \right) \\
\sigma &\sim \mathrm{Uniform}\left( 0,\infty \right)
\end{align*}

Putting it all together, the probability model for the multi-predictor linear robust regression model is:
\begin{align*}
y_n &\sim \mathrm{t}\left(\nu, \mu_n, \sigma \right) \\
\nu &\sim \mathrm{Cauchy}_+\left(7,5 \right) \\
\mu_n &= \alpha + \sum\limits_{p=1}^P X_{n,p}\beta_p \\
\alpha &\sim \mathrm{Uniform}\left( -\infty,\infty \right) \\
\beta_p &\sim \mathrm{Uniform}\left( -\infty,\infty \right), p=1,\cdots,P \\
\sigma &\sim \mathrm{Uniform}\left( 0,\infty \right)
\end{align*}

\end{document}