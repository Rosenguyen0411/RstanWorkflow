\documentclass[12pt]{article}

\usepackage[top=1in, bottom=1in, left=1in, right=1in]{geometry}
\usepackage{amsfonts}
\usepackage{amsmath}
\usepackage{amssymb}

\setlength\parindent{0pt}
\begin{document}

\title{Probability Model For A Multi-Logit Regression Model}
\author{Nan Wu \\ nanw@udel.edu}
\date{}
\maketitle

\section{Multi-Logit Regression Model}

The multi-logit regression model is given as
\begin{align}
\begin{split}
  y_n &\sim \mathrm{Categorical} \left(\mathbf{p}\right) \\
  \mathbf{p} &=\left(p_1,\ldots,p_K \right)\\
  p_k &= \frac{\exp\left( \sum\limits_{p=0}^P X_{n,p}\beta_{p,k} \right)}{\sum \limits_{k=1}^K \exp \left( \sum\limits_{p=0}^P X_{n,p}\beta_{p,k} \right)}
\end{split}
\end{align}

where $y_n$ has $K$ discrete outcomes, say $y_n \in \left\{1,\ldots,K\right\}$, $X\in \mathbb{R}^{N \times (P+1)}$ is the predictor matrix with the entries in the first column all being one, and $\beta \in \mathbb{R}^{(P+1) \times K}$ is the coefficient matrix.

\section{Probability Model}

Stan allows us to use improper priors if we don't have any prior knowledge about the parameters. As an example, we could assume the following priors for $\beta$:
\begin{align*}
\beta_{p,k} \sim \mathrm{Normal}\left( 0,5 \right)
\end{align*}

Putting it all together, the probability model for the multi-predictor logistic regression model is:
\begin{align*}
  y_n &\sim \mathrm{Categorical} \left(\mathbf{p}\right) \\
  \mathbf{p} &=\left(p_1,\ldots,p_K \right) \\
  p_k &= \frac{\exp\left( \sum\limits_{p=0}^P X_{n,p}\beta_{p,k} \right)}{\sum \limits_{k=1}^K \exp \left( \sum\limits_{p=0}^P X_{n,p}\beta_{p,k} \right)} \\
  \beta_{p,k} &\sim \mathrm{Normal}\left( 0,5 \right)
\end{align*}

\end{document}