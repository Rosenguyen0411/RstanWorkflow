\documentclass[12pt]{article}

\usepackage[top=1in, bottom=1in, left=1in, right=1in]{geometry}
\usepackage{amsfonts}
\usepackage{amsmath}
\usepackage{amssymb}

\setlength{\parindent}{0pt}
\setlength{\parskip}{0.1in}
%\setlength{\parskip}{0.1\baselineskip}

\def\eqa{\left(\sum \limits_{d=1}^D X_{n,d}\beta_{l_n,d} \right)}

\begin{document}

\title{Probability Model For A 1PL Item-response Model}
\author{Nan Wu \\ nanw@udel.edu}
\date{}
\maketitle

\section{1PL Item-response Model (Rasch Model)}

Suppose $J$ students are given a test with $K$ questions, with $y_{jk}=1$ if the response of student $j$ to question $k$ is correct, and  $y_{jk}=0$ otherwise. Assume that there are $N$ individual responses (observations) with each response $n$ associated with a person $j_n$ and a question $k_n$, where $j_n\in{1,\ldots, J}$, and $k_n\in{1, \ldots, K}$, and $\alpha_j$ is the ability of student $j$, $\beta_k$ is the difficulty of question $k$, then the 1PL (Rasch) model can be written as
\begin{equation} \label{eq:1}
  \mathrm{Pr}\left(y_n=1\right)=\mathrm{logit}^{-1}\left(\alpha_{j_n} - \beta_{k_n}\right)
\end{equation}

\section{Probability Model}

The model in equation \ref{eq:1} suffers from additive identifiability issues without proper priors, because the probabilities depend only on the relative positions of the ability and difficulty parameters. One simple way to identify the model is to use prior for either $\alpha$ or $\beta$ located at 0 (but not both).

We assume the following priors for $\alpha_j$ and $\beta_k$
\begin{align*}
  \alpha_j &\sim \mathrm{Normal}\left(\mu_\alpha,1\right)\\
  \mu_\alpha &\sim \mathrm{Normal} \left(0.75, 1\right) \\
  \beta_k &\sim \mathrm{Normal} \left(0, 1\right)
\end{align*}
where $j=1, \ldots, J$, and $k=1, \ldots, K$.

Putting it all together, we have the probability model for this 1PL (Rasch) model as:
\begin{align*}
  y_n &\sim \mathrm{Bernoulli}\left(p_n\right) \\
  p_n &= \mathrm{logit}^{-1}\left(\alpha_{j_n}-\beta_{k_n}\right) \\
  \alpha_j &\sim \mathrm{Normal\left(\mu_\alpha, 1 \right)}\\
  \mu_\alpha &\sim \mathrm{Normal} \left(0.75, 1\right) \\
  \beta_k &\sim \mathrm{Normal} \left(0, 1\right)
\end{align*}
where $j_n \in \left(1,\ldots,J \right)$, $k_n \in \left(1,\ldots,K \right)$, $j=1,\ldots, J$, and $k=1,\ldots,K$.

\end{document}